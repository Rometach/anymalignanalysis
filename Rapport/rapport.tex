\documentclass{article}
\usepackage{bookman}
\usepackage{graphicx}
\usepackage[utf8]{inputenc}
\usepackage[english,french]{babel}
\usepackage[table]{xcolor}
\usepackage{amsmath}
\usepackage{amssymb}


\title{Titre}
\author{Auteur}
\date{\today}


\begin{document}
\maketitle
\begin{abstract}
Résumé.
\end{abstract}
\newpage
\tableofcontents
\newpage


\section{Introduction}



\section{Contexte, rappels}

\subsection{Traitement des langues parlées}

\subsection{Traduction automatique}



\section{Présentation d'\emph{anymalign}}

\subsection{L'algorithme}

\subsection{Qualités et défauts}
(L'analyse déjà réalisée)


\section{Mesures d'association}

\subsection{Topo sur les mesures d'association}

\subsection{Comparaison avec anymalign}
Exploitation des graphiques


\section{Analyse théorique}

\subsection{Alignement de e et f}

\subsection{Alignement de n-grams}


\section{Conclusion}


\section{Bibliographie}


\section{Annexes}

\end{document}












%\section{Autres pistes}
%\subsection{Chinois}
%\subsection{Vitesse de convergence}
%\subsection{Discussion sur la taille du sous-corpus}
%\subsection{generateur.c}